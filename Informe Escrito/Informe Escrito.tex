\documentclass[12pt,a4paper]{article}
\usepackage[utf8]{inputenc}
\usepackage[spanish]{babel}
\usepackage{amsmath}
\usepackage{amsfonts}
\usepackage{amssymb}
\usepackage{makeidx}
\usepackage{graphicx}
\usepackage{lmodern}
\usepackage{kpfonts}
\usepackage{fourier}
\usepackage[left=2cm,right=2cm,top=2cm,bottom=2cm]{geometry}
\author{LauraMartirBeltranC113}
\begin{document}

\begin{Huge}

\begin{center}

\textbf{Havana Univerisity Language for Kompilers} (H.U.L.K.)\\\\

\textbf{Informe Escrito}

\end{center}

\begin{center}
Grupo: C-111\\
Facultad: MATCOM \\
Universidad de La Habana\\
Estudiante: Laura Mártir Beltrán
\end{center}

\end{Huge}

\begin{LARGE}

\begin{center}
\textbf{¿Qué es un intérprete de lenguaje de programación?}

\end{center}

\begin{flushleft}
Es un programa informático capaz de analizar y ejecutar otros programas. Los intérpretes solo traducen del lenguaje de programación al código de máquina, a medida que sea necesario, instrucción por instrucción y normalmente sin guardar el resultado de dicha traducción.\\
Los programas interpretados suelen ser mas lentos que los compilados debido a la necesidad de traducir el programa mientras se ejecuta, pero a cambio son más flexibles como entornos de programación y depuración ,y permiten ofrecer al programa interpretado un entorno no dependiente de la máquina donde se ejecuta el intérprete, pero si dependiente del propio intérprete.
\end{flushleft}

\begin{flushleft}
Este, está estructurado por tres procesos:\\

$\bullet$ Análisis Léxico (lexer).\\
$\bullet$ Análisis Sintáctico (parser).\\
$\bullet$ Análisis Semántico.

\end{flushleft}

\end{LARGE}

\newpage{

\begin{LARGE}

\begin{center}

\textbf{Análisis Léxico} (lexer):

\end{center}

\begin{flushleft}

Dada una entrada no vacía, lo primero que debe hacer nuestro intérprete es desglosar todos los tokens válidos (si los hay, y en caso de alguno no ser válido, se devuelve un error) que pertenecen a esta, deshaciéndose de todos los espacios en blanco.

\end{flushleft}

\begin{center}

\textbf{Análisis Sintáctico} (parser):

\end{center}

\begin{flushleft}

Dado un conjunto de reglas que definen las combinaciones de símbolos considerados declaraciones o expresiones correctamente estructuradas en el lenguaje H.U.L.K., se comenzará a analizar la sintaxis de la entrada ya tokenizada. En caso de no cumplir todas las reglas sintácticas de H.U.L.K., se devolverá un error. 

\end{flushleft}

\begin{center}

\textbf{Análisis Semántico}

\end{center}

\begin{flushleft}

De la misma forma que ocurre en el análisis sintáctico, H.U.L.K. posee un conjunto de reglas semánticas las cuales se deben cumplir en la asignacion de valores y operaciones de la entrada para devolver una salida correctamente traducida, de otra manera, se devolverá un error. 

\end{flushleft}

\end{LARGE}

}

\end{document}